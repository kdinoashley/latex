%----------------------------------------------------------------------
% This is a simple template for a document such as a project
%----------------------------------------------------------------------
\documentclass[a4paper]{article}
\pagestyle{plain}

%----------------------------------------------------------------------
% First read in some definitions (e.g. for `\mmapic')
%----------------------------------------------------------------------
\usepackage{st116latex}

\begin{document}

%----------------------------------------------------------------------
% Heading (change the format if you wish)
%----------------------------------------------------------------------
\begin{center}
\LARGE
\textbf{\huge Ashley's perfect \LaTeX\ Project 2018--2019}   % Insert title for your project
\par
\textbf{Ashley Chen}        % Insert your name
\\
\textbf{Data Science  1822440}      % Insert your degree and ID number
\end{center}

%----------------------------------------------------------------------
% First section
%----------------------------------------------------------------------
\section{Introduction}
Here is some typical text.
It could contain simple formulae like $e^{i\pi}+1=0$ as well as 
\emph{emphasised} text, \textbf{bold}, 
\textsf{sans serif} or \texttt{typewriter} fonts.
But the use of too many different fonts 
\textsc{is Ugly \large \& Distracting!}

%----------------------------------------------------------------------
% Subsection
%----------------------------------------------------------------------
\subsection{Examples of graphics}

%----------------------------------------------------------------------
% Examples of Graphics
%----------------------------------------------------------------------

Here are some graphics (in PNG format); 
the default height using \textsf{st116latex.sty} is 6cm: 
\centerpic{cont.png}

Here is the same plot with height set to 3cm: 
\centerpic[3cm]{cont.png}

You can include graphics in \emph{figures}, 
which \LaTeX\ if necessary moves to a different position in the document.
You may then refer to the figure in the main text using the commands
\verb¬\label¬, \verb¬\ref¬ and \verb¬\pageref¬.
\begin{figure}
\centerpic[4cm]{cont.png}
\label{figcont}   % optionally give the figure a label so you can refer to it
\caption{Typical (optional) caption}   % optionally add a caption
\end{figure}
For example, figure \ref{figcont} on page \pageref{figcont}
shows a 4 centimetre high contour plot.

You can also include JPEGs:
\centerpic[3cm]{babykiller.jpg}   % photo of a baby killer whale

%----------------------------------------------------------------------
% Using the picture environment
%----------------------------------------------------------------------
\subsection{Example using \LaTeX\ \texttt{picture} environment}

Strips of height 1 and lengths $1, 2, \dots, n$ 
can be arranged in a triangle,
and two such triangles fit together to make a rectangle 
with sides $n$ and $(n+1)$.
The following diagram shows this for $n=5$:
\begin{center}
\setlength{\unitlength}{0.7cm}
\begin{picture}(7,6)(-0.5,-0.5)
\linethickness{0.05mm}
\multiput(0,0)(1,0){7}{\line(0,1){5}}
\multiput(0,0)(0,1){6}{\line(1,0){6}}
\linethickness{0.7mm}
\put(0,0){\line(0,1){5}}
\put(6,0){\line(0,1){5}}
\put(0,0){\line(1,0){6}}
\put(0,5){\line(1,0){6}}
\multiput(1,1)(1,1){4}{\line(1,0){1}}
\multiput(1,0)(1,1){5}{\line(0,1){1}}
\end{picture}
\end{center}
The area of the rectangle is $n(n+1)$, 
so the area of each triangle is ${1\over2}n(n+1)$. \\
Hence $1+2+\dots+n = {1\over2}n(n+1)$.


%----------------------------------------------------------------------
% Second section
% (typically a project contains 3 to 6 sections).
%----------------------------------------------------------------------
\section{Next Section}
More stuff here
(a project typically contains 3--6 sections).

\subsection{Subsection}
Yet more stuff.
Did you know that 
\[
2\sum_{k=0}^{17}\bigl(\sin(k\pi/12)-\cos(k\pi/5)\bigr)
 = \sqrt0 +  \sqrt1 + \sqrt2 + \sqrt3 + \sqrt4 + \sqrt5 + \sqrt6\,?
\]

\subsection{Another subsection}
The equation is $\alpha x^2 + \beta x + \gamma = 0$.

%----------------------------------------------------------------------
% Conclusion
%----------------------------------------------------------------------
\section{Summary \& Conclusions}
It's usually a good idea to end with a summary of what you've found,
your conclusions, and suggestions for possible future work.

\end{document}
